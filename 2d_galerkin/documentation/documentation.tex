\documentclass[12pt]{amsart}

% PACKAGES
\usepackage[top=.5in, bottom=.5in, left=.7in, right=.7in]{geometry}
\usepackage{amsmath}        % AMS fonts and symbols
\usepackage{amsthm}         % Theorems in AMS style
\usepackage{amssymb}        % Additional AMS symbols
\usepackage{amsfonts}       % Additional AMS fonts
\usepackage{amsthm}			% Theorems, remarks, etc
\usepackage{bbm}			% Blackboard bold numbers
\usepackage{nicefrac}       % Partially stacks horizontal fractions
\usepackage{scrextend}      % Required to manipulate margins
\usepackage{mathrsfs}       % Adds \mathscr, i.e. math script style
\usepackage{hyperref}       % Displays hyperlinks
\usepackage{enumitem}       % Additional enumeration features
\usepackage{verbatim}       % Displays code-like text
\usepackage{graphicx}       % Used for graphics
\usepackage{cite}           % Enables bibtex bibliographies
\usepackage{soul}           % Special text editing, for MP comments
\usepackage{color}
  
% PAGE LAYOUT
\setlength{\parskip}{\baselineskip}
\setlength\parindent{0pt}

\begin{document}
\subsection{Nodal Quadrature}
For function $f = f(x,y)$ and triangular domain $\Delta$ defined by nodes
$(x_i,y_i)$, $1 \leq i \leq 3$, the nodal quadrature rule is given by 
%
\begin{eqnarray}
	\int_\Delta f(x,y) dx \approx \frac{1}{3} |\Delta| \sum_{i=1}^{3} f(x_i,y_i). 
\end{eqnarray}
%
This method is generally first order accurate, meaning that if $h$ is reduced
by half, then the error in the approximation is reduced by half. 


\subsection{Gaussian One-Point Quadrature}
For $f$ and $\Delta$ as above, the Gaussian one-point quadrature formula is 
%
\begin{eqnarray}
	\int_\Delta f(x,y) dx \approx |\Delta| f(x_c,y_c)
\end{eqnarray}
%
where $(x_c,y_c)$ is the centroid of $\Delta$, given by 
%
\begin{eqnarray}
	(x_c,y_c) = \frac{1}{3} \sum_{i=1}^{3} (x_i,y_i).	
\end{eqnarray}
%
This method is also generally first order accurate.


\subsection{Quadrature for Linear Functions}
For $\Delta$ as above and linear $f$, Gaussian one-point quadrature is exact.
Also, due to the linearity of $f$,
%
\begin{eqnarray}
	f(\bar{x},\bar{y}) = \frac{1}{3} \sum_{i=1}^{3} f(x_i,y_i).
\end{eqnarray}
%
so that Gaussian one-point quadrature is equivalent to nodal quadrature.


\subsection{Nodal Quadrature vs. Gaussian One-Point Quadrature}
For a single triangle, Gaussian one-point quadrature requires compuation of the
centroid of $\Delta$, followed by a single function evaluation. Nodal
quadrature requires three function evaluations. So, on a single triangle,
Gaussian one-point quadrature is less expensive. 

However, on an entire mesh of triangles, nodal quadrature is usually less
expensive. This is because nodes are reused (often multiple times) in defining
the triangles of the mesh; so the number of nodes in the mesh will generally be
smaller (perhaps much smaller) than the number of elements. Quadrature will be
done on each triangle and the results summed. So, while Gaussian one-point
quadrature requires one function evaluation per triangle, by contrast the
function can be evaluated (and stored) at each node in advance, and the results
of these nodal function evaluations can be reused as nodal quadrature is
performed on each triangle. Therefore, nodal quadrature requires one function
evaluation per node, whereas Gaussian one-point quadrature requires one
function evaluation per element; and the number of nodes in the mesh is likely
(much) smaller than the number of elements.

For example, Table \ref{tab:1} shows the number of nodes and elements in a
triangular mesh on $\Omega = (0,1) \times (0,1)$ for various values $h$. As the
table shows, for refine meshes, the number of triangles is greater than the
number of nodes. 

Furthermore, if you are integrating a Galerkin finite element solution, then
the solution's value will already be stored on the nodes. Therefore, there is
no need to even compute the function's value, since it will already be stored. 

%
\begin{table}[]
    \centering
    \begin{tabular}{c c c}
    \hline
		$h$ & nodes & elements \\
	\hline
		$2^0$		& 5    & 4    \\
		$2^{-1}$	& 12   & 14   \\
		$2^{-2}$	& 29   & 40   \\
		$2^{-3}$	& 92   & 150  \\
		$2^{-4}$	& 322  & 578  \\
		$2^{-5}$	& 1232 & 2334 \\
	\hline
    \end{tabular}
    \caption{}
    \label{tab:1}
\end{table}
%




Gaussian one-point quadrature
also requires to compute the centroid of $\Delta$. However, this computation
amounts to an arithmetic average and is therefore 


\end{document}
